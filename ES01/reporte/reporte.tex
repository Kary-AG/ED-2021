% Options for packages loaded elsewhere
\PassOptionsToPackage{unicode}{hyperref}
\PassOptionsToPackage{hyphens}{url}
%
\documentclass[
]{article}
\usepackage{lmodern}
\usepackage{amssymb,amsmath}
\usepackage{ifxetex,ifluatex}
\ifnum 0\ifxetex 1\fi\ifluatex 1\fi=0 % if pdftex
  \usepackage[T1]{fontenc}
  \usepackage[utf8]{inputenc}
  \usepackage{textcomp} % provide euro and other symbols
\else % if luatex or xetex
  \usepackage{unicode-math}
  \defaultfontfeatures{Scale=MatchLowercase}
  \defaultfontfeatures[\rmfamily]{Ligatures=TeX,Scale=1}
\fi
% Use upquote if available, for straight quotes in verbatim environments
\IfFileExists{upquote.sty}{\usepackage{upquote}}{}
\IfFileExists{microtype.sty}{% use microtype if available
  \usepackage[]{microtype}
  \UseMicrotypeSet[protrusion]{basicmath} % disable protrusion for tt fonts
}{}
\makeatletter
\@ifundefined{KOMAClassName}{% if non-KOMA class
  \IfFileExists{parskip.sty}{%
    \usepackage{parskip}
  }{% else
    \setlength{\parindent}{0pt}
    \setlength{\parskip}{6pt plus 2pt minus 1pt}}
}{% if KOMA class
  \KOMAoptions{parskip=half}}
\makeatother
\usepackage{xcolor}
\IfFileExists{xurl.sty}{\usepackage{xurl}}{} % add URL line breaks if available
\IfFileExists{bookmark.sty}{\usepackage{bookmark}}{\usepackage{hyperref}}
\hypersetup{
  pdftitle={Estructuras Discretas},
  hidelinks,
  pdfcreator={LaTeX via pandoc}}
\urlstyle{same} % disable monospaced font for URLs
\usepackage[margin=1in]{geometry}
\usepackage{color}
\usepackage{fancyvrb}
\newcommand{\VerbBar}{|}
\newcommand{\VERB}{\Verb[commandchars=\\\{\}]}
\DefineVerbatimEnvironment{Highlighting}{Verbatim}{commandchars=\\\{\}}
% Add ',fontsize=\small' for more characters per line
\usepackage{framed}
\definecolor{shadecolor}{RGB}{248,248,248}
\newenvironment{Shaded}{\begin{snugshade}}{\end{snugshade}}
\newcommand{\AlertTok}[1]{\textcolor[rgb]{0.94,0.16,0.16}{#1}}
\newcommand{\AnnotationTok}[1]{\textcolor[rgb]{0.56,0.35,0.01}{\textbf{\textit{#1}}}}
\newcommand{\AttributeTok}[1]{\textcolor[rgb]{0.77,0.63,0.00}{#1}}
\newcommand{\BaseNTok}[1]{\textcolor[rgb]{0.00,0.00,0.81}{#1}}
\newcommand{\BuiltInTok}[1]{#1}
\newcommand{\CharTok}[1]{\textcolor[rgb]{0.31,0.60,0.02}{#1}}
\newcommand{\CommentTok}[1]{\textcolor[rgb]{0.56,0.35,0.01}{\textit{#1}}}
\newcommand{\CommentVarTok}[1]{\textcolor[rgb]{0.56,0.35,0.01}{\textbf{\textit{#1}}}}
\newcommand{\ConstantTok}[1]{\textcolor[rgb]{0.00,0.00,0.00}{#1}}
\newcommand{\ControlFlowTok}[1]{\textcolor[rgb]{0.13,0.29,0.53}{\textbf{#1}}}
\newcommand{\DataTypeTok}[1]{\textcolor[rgb]{0.13,0.29,0.53}{#1}}
\newcommand{\DecValTok}[1]{\textcolor[rgb]{0.00,0.00,0.81}{#1}}
\newcommand{\DocumentationTok}[1]{\textcolor[rgb]{0.56,0.35,0.01}{\textbf{\textit{#1}}}}
\newcommand{\ErrorTok}[1]{\textcolor[rgb]{0.64,0.00,0.00}{\textbf{#1}}}
\newcommand{\ExtensionTok}[1]{#1}
\newcommand{\FloatTok}[1]{\textcolor[rgb]{0.00,0.00,0.81}{#1}}
\newcommand{\FunctionTok}[1]{\textcolor[rgb]{0.00,0.00,0.00}{#1}}
\newcommand{\ImportTok}[1]{#1}
\newcommand{\InformationTok}[1]{\textcolor[rgb]{0.56,0.35,0.01}{\textbf{\textit{#1}}}}
\newcommand{\KeywordTok}[1]{\textcolor[rgb]{0.13,0.29,0.53}{\textbf{#1}}}
\newcommand{\NormalTok}[1]{#1}
\newcommand{\OperatorTok}[1]{\textcolor[rgb]{0.81,0.36,0.00}{\textbf{#1}}}
\newcommand{\OtherTok}[1]{\textcolor[rgb]{0.56,0.35,0.01}{#1}}
\newcommand{\PreprocessorTok}[1]{\textcolor[rgb]{0.56,0.35,0.01}{\textit{#1}}}
\newcommand{\RegionMarkerTok}[1]{#1}
\newcommand{\SpecialCharTok}[1]{\textcolor[rgb]{0.00,0.00,0.00}{#1}}
\newcommand{\SpecialStringTok}[1]{\textcolor[rgb]{0.31,0.60,0.02}{#1}}
\newcommand{\StringTok}[1]{\textcolor[rgb]{0.31,0.60,0.02}{#1}}
\newcommand{\VariableTok}[1]{\textcolor[rgb]{0.00,0.00,0.00}{#1}}
\newcommand{\VerbatimStringTok}[1]{\textcolor[rgb]{0.31,0.60,0.02}{#1}}
\newcommand{\WarningTok}[1]{\textcolor[rgb]{0.56,0.35,0.01}{\textbf{\textit{#1}}}}
\usepackage{graphicx,grffile}
\makeatletter
\def\maxwidth{\ifdim\Gin@nat@width>\linewidth\linewidth\else\Gin@nat@width\fi}
\def\maxheight{\ifdim\Gin@nat@height>\textheight\textheight\else\Gin@nat@height\fi}
\makeatother
% Scale images if necessary, so that they will not overflow the page
% margins by default, and it is still possible to overwrite the defaults
% using explicit options in \includegraphics[width, height, ...]{}
\setkeys{Gin}{width=\maxwidth,height=\maxheight,keepaspectratio}
% Set default figure placement to htbp
\makeatletter
\def\fps@figure{htbp}
\makeatother
\setlength{\emergencystretch}{3em} % prevent overfull lines
\providecommand{\tightlist}{%
  \setlength{\itemsep}{0pt}\setlength{\parskip}{0pt}}
\setcounter{secnumdepth}{-\maxdimen} % remove section numbering

\title{Estructuras Discretas}
\usepackage{etoolbox}
\makeatletter
\providecommand{\subtitle}[1]{% add subtitle to \maketitle
  \apptocmd{\@title}{\par {\large #1 \par}}{}{}
}
\makeatother
\subtitle{Ejercicio Semanal 01: Matemáticas en LaTeX y \texttt{type}}
\author{Azpeitia García Karyme Ivette\\
Dorantes Perez Brando\\
Valencia Cruz Jonathan Josué}
\date{}

\begin{document}
\maketitle

\hypertarget{fuxf3rmulas-matemuxe1ticas-y-funciones-en-haskell}{%
\section{\texorpdfstring{Fórmulas matemáticas y Funciones en
\texttt{Haskell}}{Fórmulas matemáticas y Funciones en Haskell}}\label{fuxf3rmulas-matemuxe1ticas-y-funciones-en-haskell}}

\hypertarget{uxe1rea-de-un-circulo}{%
\subsection{Área de un Circulo}\label{uxe1rea-de-un-circulo}}

\hypertarget{fuxf3rmula-matemuxe1tica}{%
\subsubsection{Fórmula matemática}\label{fuxf3rmula-matemuxe1tica}}

La fórmula para obtener el área de un circulo denotada por:

\[
A= \pi \cdot r^2
\] donde \(r=\) al radio del circulo.

\hypertarget{funciuxf3n-en-haskell}{%
\subsubsection{\texorpdfstring{Función en
\texttt{Haskell}}{Función en Haskell}}\label{funciuxf3n-en-haskell}}

\begin{Shaded}
\begin{Highlighting}[]
\NormalTok{-- | Función que regresa el área de un círculo.}
\NormalTok{areaCirc :: Double -> Double}
\NormalTok{areaCirc x = pi*(x**}\DecValTok{2}\NormalTok{)}
\end{Highlighting}
\end{Shaded}

\hypertarget{distancia-entre-dos-puntos}{%
\subsection{Distancia entre dos
puntos}\label{distancia-entre-dos-puntos}}

\hypertarget{fuxf3rmula-matemuxe1tica-1}{%
\subsubsection{Fórmula matemática}\label{fuxf3rmula-matemuxe1tica-1}}

La fórmula para obtener la distancia entre dos puntos dados por sus
cordenadas es:

\[
d =\sqrt{(x_2 - x_1)^2 + (y_2 - y_1)^2}
\]

\hypertarget{funciuxf3n-en-haskell-1}{%
\subsubsection{\texorpdfstring{Función en
\texttt{Haskell}}{Función en Haskell}}\label{funciuxf3n-en-haskell-1}}

\begin{Shaded}
\begin{Highlighting}[]
\NormalTok{-- | Función que regresa la distancia entre dos puntos (x1, y1), (x2. y2).}
\NormalTok{distancia :: Abscisa -> Ordenada -> Abscisa -> Ordenada -> Parordenado}
\NormalTok{distancia x1 y1 x2 y2 = sqrt((x1-x2)^}\DecValTok{2}\NormalTok{+(y1-y2)^}\DecValTok{2}\NormalTok{)}
\end{Highlighting}
\end{Shaded}

\hypertarget{suma-de-gauss}{%
\subsection{Suma de Gauss}\label{suma-de-gauss}}

\hypertarget{fuxf3rmula-matemuxe1tica-2}{%
\subsubsection{Fórmula matemática}\label{fuxf3rmula-matemuxe1tica-2}}

\[
\sum_{k=1}^{n}k=\frac{n(n+1)}{2} 
\]

\hypertarget{funciuxf3n-en-haskell-2}{%
\subsubsection{\texorpdfstring{Función en
\texttt{Haskell}}{Función en Haskell}}\label{funciuxf3n-en-haskell-2}}

\begin{Shaded}
\begin{Highlighting}[]
\NormalTok{-- | Función que calcula la suma de los primeros n números (Suma de Gauss).}
\NormalTok{sumaGauss :: Int -> Int}
\NormalTok{sumaGauss x  = ((x*(}\DecValTok{1}\NormalTok{+x))`div`}\DecValTok{2}\NormalTok{)}
\end{Highlighting}
\end{Shaded}

\hypertarget{uxe1rea-del-triangulo}{%
\subsection{Área del Triangulo}\label{uxe1rea-del-triangulo}}

\hypertarget{fuxf3rmula-matemuxe1tica-3}{%
\subsubsection{Fórmula matemática}\label{fuxf3rmula-matemuxe1tica-3}}

La fórmula general para calcular el área de un triángulo es:

\[
A = \frac{b\cdot h}{2}
\]

\hypertarget{funciuxf3n-en-haskell-3}{%
\subsubsection{\texorpdfstring{Función en
\texttt{Haskell}}{Función en Haskell}}\label{funciuxf3n-en-haskell-3}}

\begin{Shaded}
\begin{Highlighting}[]
\NormalTok{-- | Función que calcula el área de un triángulo dados tres puntos.}
\NormalTok{areaTri :: Double -> Double -> Double -> Double -> Double  -> Double -> Double}
\NormalTok{areaTri x1 y1 x2 y2 x3 y3  = (((x1*y2)+(x2*y3)+(x3*y1))-((x1*y3)+(x3*y2)+(x2*y1)))/ (}\DecValTok{2}\NormalTok{)}
\end{Highlighting}
\end{Shaded}

\hypertarget{dudas}{%
\section{Dudas}\label{dudas}}

\begin{enumerate}
\def\labelenumi{\arabic{enumi}.}
\item
  ¿Cómo se utiliza el input en haskell?
\item
  ¿Qué son las mónadas?
\end{enumerate}

\hypertarget{extra}{%
\section{Extra}\label{extra}}

\begin{Shaded}
\begin{Highlighting}[]
\NormalTok{type Radio = Float}
\NormalTok{type Lado = Float}

\NormalTok{data Figura = Circulo Radio}
\NormalTok{            | Cuadrado Lado}
\NormalTok{            | Rectangulo Lado Lado}
\NormalTok{            | Punto}
\NormalTok{              deriving Show}
            
\NormalTok{perimetro :: Figura -> Float}
\NormalTok{perimetro (Circulo radio) = }\DecValTok{2}\NormalTok{ * pi * radio}
\NormalTok{perimetro (Cuadrado lado) = }\DecValTok{4}\NormalTok{ * lado}
\NormalTok{perimetro (Rectangulo ancho alto) = }\DecValTok{2}\NormalTok{ * ancho + }\DecValTok{2}\NormalTok{ * alto}
\NormalTok{perimetro (Punto) = error }\StringTok{"no se puede calcular el perimetro del punto"}
\end{Highlighting}
\end{Shaded}

Pseudocódigo

\begin{enumerate}
\def\labelenumi{(\arabic{enumi})}
\item
  Se asignan dos \texttt{type}, \texttt{Radio} y \texttt{Lado} los dos
  serán Float.
\item
  Se define el tipo de dato \texttt{Figura} el cual tendrá 4 posibles
  valores \emph{Circulo, Cuadrado, Rectangulo} y \emph{Punto} cada uno
  \texttt{Radio} o \texttt{Lado} dependiendo que sea necesarío para
  calcular su perimetro.
\item
  Como se esta dando una forma predefinida se utiliza
  \texttt{deriving\ Show}
\item
  Se declará la función \texttt{perimetro} la cual recibe un data figura
  y regresa un \texttt{Float} que indica el perimetro.
\item
  Se dan las diferentes opciones de usar \texttt{perimetro} de acuerdo
  al data figura que se ingrese.
\end{enumerate}

\end{document}
