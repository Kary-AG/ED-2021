% Options for packages loaded elsewhere
\PassOptionsToPackage{unicode}{hyperref}
\PassOptionsToPackage{hyphens}{url}
%
\documentclass[
]{article}
\usepackage{lmodern}
\usepackage{amssymb,amsmath}
\usepackage{ifxetex,ifluatex}
\ifnum 0\ifxetex 1\fi\ifluatex 1\fi=0 % if pdftex
  \usepackage[T1]{fontenc}
  \usepackage[utf8]{inputenc}
  \usepackage{textcomp} % provide euro and other symbols
\else % if luatex or xetex
  \usepackage{unicode-math}
  \defaultfontfeatures{Scale=MatchLowercase}
  \defaultfontfeatures[\rmfamily]{Ligatures=TeX,Scale=1}
\fi
% Use upquote if available, for straight quotes in verbatim environments
\IfFileExists{upquote.sty}{\usepackage{upquote}}{}
\IfFileExists{microtype.sty}{% use microtype if available
  \usepackage[]{microtype}
  \UseMicrotypeSet[protrusion]{basicmath} % disable protrusion for tt fonts
}{}
\makeatletter
\@ifundefined{KOMAClassName}{% if non-KOMA class
  \IfFileExists{parskip.sty}{%
    \usepackage{parskip}
  }{% else
    \setlength{\parindent}{0pt}
    \setlength{\parskip}{6pt plus 2pt minus 1pt}}
}{% if KOMA class
  \KOMAoptions{parskip=half}}
\makeatother
\usepackage{xcolor}
\IfFileExists{xurl.sty}{\usepackage{xurl}}{} % add URL line breaks if available
\IfFileExists{bookmark.sty}{\usepackage{bookmark}}{\usepackage{hyperref}}
\hypersetup{
  pdftitle={Estructuras Discretas},
  hidelinks,
  pdfcreator={LaTeX via pandoc}}
\urlstyle{same} % disable monospaced font for URLs
\usepackage[margin=1in]{geometry}
\usepackage{color}
\usepackage{fancyvrb}
\newcommand{\VerbBar}{|}
\newcommand{\VERB}{\Verb[commandchars=\\\{\}]}
\DefineVerbatimEnvironment{Highlighting}{Verbatim}{commandchars=\\\{\}}
% Add ',fontsize=\small' for more characters per line
\usepackage{framed}
\definecolor{shadecolor}{RGB}{248,248,248}
\newenvironment{Shaded}{\begin{snugshade}}{\end{snugshade}}
\newcommand{\AlertTok}[1]{\textcolor[rgb]{0.94,0.16,0.16}{#1}}
\newcommand{\AnnotationTok}[1]{\textcolor[rgb]{0.56,0.35,0.01}{\textbf{\textit{#1}}}}
\newcommand{\AttributeTok}[1]{\textcolor[rgb]{0.77,0.63,0.00}{#1}}
\newcommand{\BaseNTok}[1]{\textcolor[rgb]{0.00,0.00,0.81}{#1}}
\newcommand{\BuiltInTok}[1]{#1}
\newcommand{\CharTok}[1]{\textcolor[rgb]{0.31,0.60,0.02}{#1}}
\newcommand{\CommentTok}[1]{\textcolor[rgb]{0.56,0.35,0.01}{\textit{#1}}}
\newcommand{\CommentVarTok}[1]{\textcolor[rgb]{0.56,0.35,0.01}{\textbf{\textit{#1}}}}
\newcommand{\ConstantTok}[1]{\textcolor[rgb]{0.00,0.00,0.00}{#1}}
\newcommand{\ControlFlowTok}[1]{\textcolor[rgb]{0.13,0.29,0.53}{\textbf{#1}}}
\newcommand{\DataTypeTok}[1]{\textcolor[rgb]{0.13,0.29,0.53}{#1}}
\newcommand{\DecValTok}[1]{\textcolor[rgb]{0.00,0.00,0.81}{#1}}
\newcommand{\DocumentationTok}[1]{\textcolor[rgb]{0.56,0.35,0.01}{\textbf{\textit{#1}}}}
\newcommand{\ErrorTok}[1]{\textcolor[rgb]{0.64,0.00,0.00}{\textbf{#1}}}
\newcommand{\ExtensionTok}[1]{#1}
\newcommand{\FloatTok}[1]{\textcolor[rgb]{0.00,0.00,0.81}{#1}}
\newcommand{\FunctionTok}[1]{\textcolor[rgb]{0.00,0.00,0.00}{#1}}
\newcommand{\ImportTok}[1]{#1}
\newcommand{\InformationTok}[1]{\textcolor[rgb]{0.56,0.35,0.01}{\textbf{\textit{#1}}}}
\newcommand{\KeywordTok}[1]{\textcolor[rgb]{0.13,0.29,0.53}{\textbf{#1}}}
\newcommand{\NormalTok}[1]{#1}
\newcommand{\OperatorTok}[1]{\textcolor[rgb]{0.81,0.36,0.00}{\textbf{#1}}}
\newcommand{\OtherTok}[1]{\textcolor[rgb]{0.56,0.35,0.01}{#1}}
\newcommand{\PreprocessorTok}[1]{\textcolor[rgb]{0.56,0.35,0.01}{\textit{#1}}}
\newcommand{\RegionMarkerTok}[1]{#1}
\newcommand{\SpecialCharTok}[1]{\textcolor[rgb]{0.00,0.00,0.00}{#1}}
\newcommand{\SpecialStringTok}[1]{\textcolor[rgb]{0.31,0.60,0.02}{#1}}
\newcommand{\StringTok}[1]{\textcolor[rgb]{0.31,0.60,0.02}{#1}}
\newcommand{\VariableTok}[1]{\textcolor[rgb]{0.00,0.00,0.00}{#1}}
\newcommand{\VerbatimStringTok}[1]{\textcolor[rgb]{0.31,0.60,0.02}{#1}}
\newcommand{\WarningTok}[1]{\textcolor[rgb]{0.56,0.35,0.01}{\textbf{\textit{#1}}}}
\usepackage{longtable,booktabs}
% Correct order of tables after \paragraph or \subparagraph
\usepackage{etoolbox}
\makeatletter
\patchcmd\longtable{\par}{\if@noskipsec\mbox{}\fi\par}{}{}
\makeatother
% Allow footnotes in longtable head/foot
\IfFileExists{footnotehyper.sty}{\usepackage{footnotehyper}}{\usepackage{footnote}}
\makesavenoteenv{longtable}
\usepackage{graphicx,grffile}
\makeatletter
\def\maxwidth{\ifdim\Gin@nat@width>\linewidth\linewidth\else\Gin@nat@width\fi}
\def\maxheight{\ifdim\Gin@nat@height>\textheight\textheight\else\Gin@nat@height\fi}
\makeatother
% Scale images if necessary, so that they will not overflow the page
% margins by default, and it is still possible to overwrite the defaults
% using explicit options in \includegraphics[width, height, ...]{}
\setkeys{Gin}{width=\maxwidth,height=\maxheight,keepaspectratio}
% Set default figure placement to htbp
\makeatletter
\def\fps@figure{htbp}
\makeatother
\setlength{\emergencystretch}{3em} % prevent overfull lines
\providecommand{\tightlist}{%
  \setlength{\itemsep}{0pt}\setlength{\parskip}{0pt}}
\setcounter{secnumdepth}{-\maxdimen} % remove section numbering

\title{Estructuras Discretas}
\usepackage{etoolbox}
\makeatletter
\providecommand{\subtitle}[1]{% add subtitle to \maketitle
  \apptocmd{\@title}{\par {\large #1 \par}}{}{}
}
\makeatother
\subtitle{Práctica 04: Interpretaciones de la lógica}
\author{``Karyme I. Azpeitia García''\\
``Dorantes Perez Brando''\\
``Valencia Cruz Jonathan Josué''}
\date{10/31/2020}

\begin{document}
\maketitle

\hypertarget{puedes-decir-a-cuxfaales-los-siguientes-ejemplos-corresponden-nuestros-operadores}{%
\subsubsection{¿Puedes decir a cúales los siguientes ejemplos
corresponden nuestros
operadores?}\label{puedes-decir-a-cuxfaales-los-siguientes-ejemplos-corresponden-nuestros-operadores}}

Es decir qué operador asocia a donde y cuál de los siguientes cuatro
ejemplos corresponde a nuestra implementación.

\begin{longtable}[]{@{}lllll@{}}
\toprule
\(P Q R\) & \((P\wedge Q)\lor R\) & valor & \(P\wedge (Q\wedge R)\) &
valor\tabularnewline
\midrule
\endhead
\(0 0 1\) & \((0)\lor 1\) & \(1\) & \(0\wedge (1)\) &
\(0\)\tabularnewline
\bottomrule
\end{longtable}

Dado que nuestros operadores lógicos \(\wedge, \lor\) tienen la
siguiente firma

\begin{Shaded}
\begin{Highlighting}[]
\CommentTok{-- | conj, disy. Metodo que representa las compuertas AND, OR                 }
\NormalTok{  (}\OperatorTok{.^.}\NormalTok{),}\OtherTok{ (.|.) ::}\NormalTok{ a }\OtherTok{->}\NormalTok{ a }\OtherTok{->}\NormalTok{ a                                                   }
  \KeywordTok{infixl} \DecValTok{7} \OperatorTok{.^.}\NormalTok{, }\OperatorTok{.|.}
\end{Highlighting}
\end{Shaded}

Notamos que \(\wedge, \lor\) tienen la misma precedencía y ambos asocian
empezando a la izquierda por lo que el ejemplo que corresponde a nuestra
implementación es \((P\wedge Q)\lor R\).

\begin{longtable}[]{@{}lllll@{}}
\toprule
\(P Q R\) & \((P\rightarrow Q)\rightarrow R\) & valor &
\(P\wedge(Q\lor R)\) & valor\tabularnewline
\midrule
\endhead
\(000\) & \(1\rightarrow 0\) & \(0\) & \(0\rightarrow 1\) &
1\tabularnewline
\bottomrule
\end{longtable}

Dado que nuestro operador lógico \(\rightarrow\) tiene la siguiente
firma

\begin{Shaded}
\begin{Highlighting}[]
\OtherTok{(.->.) ::} \DataTypeTok{MyBool} \OtherTok{->} \DataTypeTok{MyBool} \OtherTok{->} \DataTypeTok{MyBool} \CommentTok{--ex                                       }
\KeywordTok{infixr} \DecValTok{2} \OperatorTok{.->.}                                                                   
\NormalTok{(}\OperatorTok{.->.}\NormalTok{) p q }\OtherTok{=}\NormalTok{ (}\OperatorTok{.|.}\NormalTok{) q ((}\OperatorTok{.}\NormalTok{¬}\OperatorTok{.}\NormalTok{)p) }
\end{Highlighting}
\end{Shaded}

Notamos que \(\rightarrow\) tiene precedencía \(2\) y asocia a la
derecha por lo que el ejemplo que corresponde a nuestra implementación
es \(P\wedge(Q\lor R)\).

\hypertarget{puedes-ver-por-quuxe9-pwedge-rrightarrow-q-se-evaluxfaa-a-btrue}{%
\subsubsection{\texorpdfstring{¿Puedes ver por qué
\(P\wedge R\rightarrow Q\) se evalúa a
BTrue?}{¿Puedes ver por qué P\textbackslash wedge R\textbackslash rightarrow Q se evalúa a BTrue?}}\label{puedes-ver-por-quuxe9-pwedge-rrightarrow-q-se-evaluxfaa-a-btrue}}

\begin{Shaded}
\begin{Highlighting}[]
\NormalTok{modelo }\OtherTok{=}\NormalTok{[}\StringTok{"P"}\NormalTok{,}\StringTok{"Q"}\NormalTok{]}
\NormalTok{f}\OtherTok{=}\NormalTok{(}\DataTypeTok{Impl}\NormalTok{(}\DataTypeTok{Conj}\NormalTok{ (}\DataTypeTok{Var} \StringTok{"P"}\NormalTok{)(}\DataTypeTok{Var} \StringTok{"R"}\NormalTok{))(}\DataTypeTok{Var} \StringTok{"Q"}\NormalTok{))}
\end{Highlighting}
\end{Shaded}

Sabemos que \texttt{"P",\ "Q"} tienen valor \(1\) y dado que en nuestra
implementación \(\wedge\) tiene mayor precedencía que \(\rightarrow\) y
asocia por la izquierda entonces la fórmula se representa
\((P\wedge R)\rightarrow Q\), de tal manera que:

\begin{center}

$P\wedge R$ se evalua en $BFalse$

$BFalse\rightarrow Q$ se evalúa en ```BTrue```

\end{center}

\hypertarget{funciona-nuestra-implementaciuxf3n}{%
\subsubsection{¿Funciona nuestra
implementación?}\label{funciona-nuestra-implementaciuxf3n}}

El siguiente acertijo fue propuesto por Lewis Carroll:

\begin{itemize}
\item
  Todos los bebés son ilógicos.
\item
  Nadie que sea despistado puede manejar un cocodrilo.
\item
  Las personas ilógicas son despistadas.
\end{itemize}

¿Qué se puede concluir del argumento?

\begin{center}

Los bebés no pueden manejar un cocodrilo.
\end{center}

Variables proposicionales:

\begin{itemize}
\item
  B: Es un bebé.
\item
  M: Puede manejar un cocodrilo.
\item
  L: Es lógico.
\item
  D: Es despistado.
\end{itemize}

Usando las variables proposicionales anteriores tenemos

\begin{center}

\textit{Todos los bebés son ilógicos.}

$B\rightarrow \neg L$

\textit{Nadie que sea despistado puede manejar un cocodrilo.}

$D\rightarrow\neg M$

\textit{Las personas ilógicas son despistadas.}

$\neg L\rightarrow\neg D$

\textit{Los bebés no pueden manejar un cocodrilo.}

$B\rightarrow\neg M$

\end{center}

Por lo anterior podemos ver que tenemos como premisas
\(B\rightarrow \neg L\),\(D\rightarrow\neg M\),\(\neg L\rightarrow\neg D\)
y como conclusión \(B\rightarrow\neg M\).

Ahora veamos que nuestro argumento es correcto, es decir

\[
\models((B\rightarrow \neg L)\wedge (D\rightarrow\neg M)\wedge(\neg L\rightarrow\neg D))\rightarrow (B\rightarrow\neg M) 
\] Ahora planteamos nuestro módelo \(\mathcal{M}\) con
\(B=0, M=0, L=1, D=1\), para comprobar que nuestro modelo es correcto
utilizaremos la fúnción \texttt{ìnterpreta}, donde nuestro módelo es
representado con \texttt{{[}"M","L"{]}} y la expresión
\(((B\rightarrow \neg L)\wedge (D\rightarrow\neg M)\wedge(\neg L\rightarrow\neg D))\rightarrow (B\rightarrow\neg M)\)
la llamaremos \(e\).

\begin{Shaded}
\begin{Highlighting}[]
\OperatorTok{*}\DataTypeTok{Formulas}\OperatorTok{>}\NormalTok{ interpreta [ }\StringTok{"L"}\NormalTok{,}\StringTok{"D"}\NormalTok{] e}
\PreprocessorTok{#t}
\end{Highlighting}
\end{Shaded}

Entonces podemos decir que nuestro argumento es correcto.

\end{document}
